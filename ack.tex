
%%% Local Variables:
%%% mode: latex
%%% TeX-master: "../main"
%%% End:

\begin{ack}
	朝雨浥去轻尘,窗外柳色正新。樱花第十次绽放,爱校路又一次挤满了即将远行的留影人。只是这一季的毕业,终于到了我们。
	十年前,弱冠懵懂,踏着同济百年校庆的灿烂而来;十年后,已近而立,见证了同济十年的辉煌。
	春夏秋冬四季景,留在同济的不仅是美好的记忆;本硕博十载青春,放在时光里的是道不尽的感激与不舍。
	如今即将博士毕业,回首在上海的这十年,同济俨然已经成了第二个家。
	从新奇到熟悉,从熟悉到留恋,而这一次的离开又如同记忆中十八岁时的远行,留恋却又无奈。

	在这里我首先想感谢我的导师程玖兵教授。您教我成长,诲我做人,在从大三至今已将有八年矣。
	从概念到理论,从编程到撰文,在科研中每一处细节都用精益求精的态度来要求自己和我们。
	解惑时您耳提面命,懈怠时您谆谆教诲,迷茫时您指明方向,如今博士论文已行将完成,其中每一页里面都包含了您的心血与付出。
	明师之恩,重于父母多矣,无以为报,当以更大的努力向您看齐。

	感谢地震组这个大家庭在研究生期间带给我的美好回忆。感谢马在田院士,先生之风,山高水长,是您的努力让今天的地震组人才济济。
	感谢耿建华教授的岩石物理储层的课程,感谢王华忠教授的成像与反演课程以及信号处理等课程,感谢董良国教授的地震波传播与成像课程,感谢杨锴教授的地震
	勘探原理,感谢钟广法教授的地震地质课程,感谢刘堂晏教授测井以及岩石物理课程
	感谢刘玉柱副教授的理论反演课程,感谢赵峦啸和屠宁老师在组会讨论中的指点与帮助,你们的意见与建议对我帮助良多。

	感谢地震组里的兄弟姐妹同门之谊。
	在这里,我还要感谢同济,感谢海院和地震组给了我港湾,是你们百年树人的努力让我有了远航的技能和信心。

	樱花灿烂,终会凋零;胜地久长,盛筵难再。
	然而,有风有雨才有绮丽的彩虹,
	有离别才会显得相聚的珍贵,人生总是在不停的变换中才会丰富多彩。


\end{ack}
