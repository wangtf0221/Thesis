\chapter{共轭状态法推导弹性波WERTI的梯度}
\label{cha:AdjointForEWERTI}
本节中,我们采用Plessix(2006)给出的共轭状态法导出弹性波WERTI的梯度。我们共有三个状态方程其中两个为式\eqref{eq:WE}
和\eqref{eq:DeltaWE}。第三个状态方程来自DIW的优化约束。

我们知道DIW是要找出$\mathbf{\tau}=\tau(\mathbf{x}_r,t;\mathbf{x}_s)$,使得目标函数$D(l)$最小,其中$D(l)$为:
\begin{equation}
	D(\tau)=\frac{1}{2}\int
	d\mathbf{x}_sd\mathbf{x}_r(d_c(\mathbf{x}_r,t;\mathbf{x}_s)-
	d_o(\mathbf{x}_r,t+\tau(\mathbf{x}_r,t;\mathbf{x}_s);\mathbf{x}_s))^2
        \label{eq:Dl}
\end{equation}
而目标函数最小则需要满足:

\begin{equation}
	\frac{\partial D}{\partial \tau}=\int
	d\mathbf{x}_sd\mathbf{x}_ru(\mathbf{x}_r,t;\mathbf{x}_s)=0.
        \label{eq:PartialD}
\end{equation}
其中$u(\mathbf{x}_r,t;\mathbf{x}_s)=\dot{d}_o(\mathbf{x}_r,t+\tau;\mathbf{x}_s)(d_o(\mathbf{x}_r,t+\tau;\mathbf{x}_s)-
d_c(\mathbf{x}_r,t;\mathbf{x}_s))$
偏导数$\frac{\partial D}{\partial
\tau}$一定为0,如果不为0,则我们总能找到一个$\tau$使得$D$变得更小,而这与DIW找到的最小值相矛盾。则式\eqref{eq:Partial}为
第三个状态方程。我们可以定义增广函数:


