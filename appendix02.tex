\chapter{共轭状态法推导弹性波WERTI的梯度}
\label{cha:AdjointForEWERTI}
本节中,我们采用Plessix(2006)给出的共轭状态法导出弹性波WERTI的梯度。我们共有三个状态方程其中两个为式\eqref{eq:WE}
和\eqref{eq:DeltaWE}。第三个状态方程来自DIW的优化约束。

我们知道DIW是要找出$\mathbf{\tau}=\tau(\mathbf{x}_r,t;\mathbf{x}_s)$,使得目标函数$D(l)$最小,其中$D(l)$为:
\begin{equation}
	D(\tau)=\frac{1}{2}\int
	d\mathbf{x}_sd\mathbf{x}_r(d_c(\mathbf{x}_r,t;\mathbf{x}_s)-
	d_o(\mathbf{x}_r,t+\tau(\mathbf{x}_r,t;\mathbf{x}_s);\mathbf{x}_s))^2
        \label{eq:Dl}
\end{equation}
而目标函数最小则需要满足:
\begin{equation}
	\frac{\partial D}{\partial \tau}=\int
	d\mathbf{x}_sd\mathbf{x}_r\alpha(\mathbf{x}_r,t;\mathbf{x}_s)=0.
        \label{eq:PartialD}
\end{equation}
其中$\alpha(\mathbf{x}_r,t;\mathbf{x}_s)=\dot{d}_o(\mathbf{x}_r,t+\tau;\mathbf{x}_s)(d_o(\mathbf{x}_r,t+\tau;\mathbf{x}_s)-
d_c(\mathbf{x}_r,t;\mathbf{x}_s))$
偏导数$\frac{\partial D}{\partial
\tau}$一定为0,如果不为0,则我们总能找到一个$\tau$使得$D$变得更小,而这与DIW找到的最小值相矛盾。则式\eqref{eq:PartialD}为
第三个状态方程。我们可以定义增广函数:
\begin{equation}
	\begin{split}
	\mathcal{L}(\mathbf{\tau},\mathbf{u},\mathbf{\psi},\mathbf{\mu},\bar{\mathbf{u}},\bar{\mathbf{\psi}})
	&=\frac{1}{2}\int \tau^2(\mathbf{x}_r,t;\mathbf{x}_s)d\mathbf{x}_sd\mathbf{x}_rdt\\
	&-\int
	\mu(\mathbf{x}_r,t;\mathbf{x}_s)\alpha(\mathbf{x}_r,t;\mathbf{x}_s)d\mathbf{x}_sd\mathbf{x}_rdt\\
	&-\int \bar{{u_i}}\left(\rho\frac{\partial^2
	\psi_i }{\partial t^2}-\frac{\partial}{\partial x_j}(c_{ijkl}\frac{\partial
	\psi_k}{\partial x_l})-\frac{\partial}{\partial x_j}(\delta
	c_{ijkl}\frac{\partial u_k}{\partial x_l})\right)d\mathbf{x}_sd\mathbf{x}_rdt\\
	&-\int \bar{{\psi_i}}\left(\rho\frac{\partial^2 u_i }{\partial
	t^2}-\frac{\partial}{\partial x_j}(c_{ijkl}\frac{\partial u_k}{\partial x_l}) -
	f_i\right)d\mathbf{x}_sd\mathbf{x}_rdt,
	\end{split}
        \label{eq:Lagrangian}
\end{equation}
其中,$\tau(\mathbf{x}_r,t;\mathbf{x}_s)$,$\mathbf{u}$和$\mathbf{\psi}$为状态变量,同时$\mu(\mathbf{x}_r,t;\mathbf{x}_s)$,$\bar{\mathbf{u}}$和$\bar{\mathbf{\psi}}$分别为伴随状态变量,$\bar{\mathbf{u}}$和$\bar{\mathbf{\psi}}$也即反传波场。

伴随状态变量可以通过使增广函数对状态变量偏导数为零来求取,也即使得$\frac{\partial \mathcal{L}}{\partial \mathbf{\tau}}=0$,
$\frac{\partial \mathcal{L}}{\partial \mathbf{u}}=0$和$\frac{\partial \mathcal{L}}{\partial \mathbf{\psi}}=0$。
则
\begin{equation}
	\frac{\partial \mathcal{L}}{\partial \mathbf{\tau}}=\tau(\mathbf{x}_r,t;\mathbf{x}_s)
	-\mu(\mathbf{x}_r,t;\mathbf{x}_s)\frac{\partial \alpha(\mathbf{x}_r,t;\mathbf{x}_s)}{\partial \tau(\mathbf{x}_r,t;\mathbf{x}_s)},
        \label{eq:PartialTau}
\end{equation}
可得
\begin{equation}
	\mu(\mathbf{x}_r,t;\mathbf{x}_s)=\frac{\tau(\mathbf{x}_r,t;\mathbf{x}_s)}{ h_i(\mathbf{x}_r,t;\mathbf{x}_s)},
        \label{eq:AdjointTau}
\end{equation}
其中$h_i(\mathbf{x_r},t;\mathbf{x_s})=(\dot{d}^o_i(\mathbf{x_r},t+\tau;\mathbf{x_s}))^2-\ddot{d}^o_i(\mathbf{x_r},t+\tau;\mathbf{x_s})
(d^c_i(\mathbf{x_r},t;\mathbf{x_s})-d^o_i(\mathbf{x_r},t+\tau;\mathbf{x_s}))$。而$\dot{d}$则代表$d$在时间方向的导数。同时,通过另外两个偏导数方程可得:
\begin{equation}
	\rho\frac{\partial^2 \bar{u}_i }{\partial t^2}
	-\frac{\partial}{\partial x_j}(c_{ijkl}\frac{\partial \bar{u}_k}{\partial x_l})=\mu(\mathbf{x}_r,t;\mathbf{x}_s)
	\dot{d}^o_i(\mathbf{x_r},t+\tau;\mathbf{x_s}),
        \label{eq:PartialU}
\end{equation}
和
\begin{equation}
	\rho\frac{\partial^2 \bar{\psi}_i }{\partial t^2}
	-\frac{\partial}{\partial x_j}(c_{ijkl}\frac{\partial \bar{\psi}_k}{\partial x_l})=
	\frac{\partial}{\partial x_j}(\delta c_{ijkl}\frac{\partial \bar{u}_k}{\partial x_l}).
        \label{eq:PartialPsi}
\end{equation}
这样就可以求得梯度:
\begin{equation}
	\frac{\partial \mathcal{L}}{\partial c_{ijkl}}=
    \frac{\partial E}{\partial c_{ijkl}}=-\int (\frac{\partial u_{i}}{\partial
    x_j}\frac{\partial \bar{\psi}_{k}}{\partial x_l}+\frac{\partial \bar{u}_{i}}{\partial
    x_j}\frac{\partial \psi_{k}}{\partial x_l}),
    \label{eq:GradientCijkl1}
\end{equation}

