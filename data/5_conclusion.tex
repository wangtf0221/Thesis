\chapter{结论与展望}
\section{结论}
为了恢复弹性介质模型中的不同波数成分,本文分别从EFWI,E-WERTI以及E-LSRTM三个方面入手来展开研究。
本文针对弹性波中的反演问题,以模式解耦作为获取数据子集以及预条件梯度的主要手段,对EFWI,E-WERTI以及E-LSRTM中各自存在的问题设计了不同的
反演策略,从而降低其反演中的参数耦合,非线性等问题。通过文中的研究,主要得到了以下几点认识与结论:

(1)多参数反演中,不同参数扰动产生的相同模式的数据扰动会造成EFWI中的参数耦合效应。
采用弹性波模式解耦对EFWI梯度进行预条件可以降低EFWI中的参数耦合程度。该预条件可以通过正传波场与解耦之后反传波场的互相关快速实现。
Hessian与分辨率矩阵的分析指出采用S波数据对$V_s$进行反演,在初始模型足够好的条件下,可以认为是单参数反演过程,
同时也证明解耦正传波场对梯度进行预条件并不能进一步压制EFWI中的参数耦合效果。

(2)密度反演会增加EFWI的非线性程度,尤其是在软海底环境中PS波非常弱的情况下。模式解耦对压制密度与$V_p$和$V_s$间的耦合作用有限,密度的反演仍然是
十分困难的课题,可能需要更多地加入多尺度策略甚至Hessian的信息。但是引入密度变化也并不十分影响模式解耦压制$V_p$与$V_s$间耦合的效果。

(3)WERTI采用走时信息作为目标函数,可以与模型背景速度间建立更加线性的关系。相比互相关提取走时的方式,DIW方式在模型复杂的情况下更加可靠稳健。

(4)在反射波路径的计算中,PP,SP,以及PS的互相关模式对$V_s$的反射波路径都有着较强的干扰作用,只有SS模式能够代表PS波中右半支的S波路径。
因此地面数据分离与空间波场解耦可以将背景$V_p$与$V_s$的反演分解为两个独立的步骤,也即先利用PP波反演$V_p$,后利用PS波反演$V_s$。

(5)在E-LSRTM中Born数据并不能直接匹配反射波数据,尤其是在偏移模型中包含较多高频信息时需要将反射数据与背景场数据做差才能与满足Born近似
下的数据残差收敛。E-LSRTM可以提高成像分辨率并部分压制参数耦合,但是无法做到完全压制。而模式解耦的预条件方式可以非常有效地压制$\delta
V_s$反演中来自$\delta V_p$的影响。
在参数耦合程度较弱时,在模式解耦帮助下进行双参数同时反演可以快速有效地获得$\delta
V_s$与$\delta V_p$的影响。而在参数耦程度较强时,可以通过先反演$\delta V_s$后反演$\delta
V_p$来进一步压制$\delta V_p$中来自$\delta V_s$耦合效应干扰。
\section{成果与创新点}
本文中围绕模式解耦工具,针对不同反演方法通过分析指导反演策略的设计,创新点主要有以下几个方面:

(1)在EFWI问题中,文中通过分析模式解耦的偏导数波场,解释了模式解耦压制EFWI中参数耦合的物理机制。
%在此过程中,梯度与Hessian计算时的交叉项可以近似为0,也即
%不同波模式的偏导数波场之间及其与不同波模式数据之间的互相关的能量要远小于相同波模式间的互相关。
基于交叉项近似,可以采用解耦反传波场而非地面数据来对EFWI的梯度进行预条件,这样避免了地面数据的分离,并可以进行双参数的同时反演。
此外,从模式解耦出发分析Hessian矩阵以及分辨率矩阵,进而发现在初始模型足够好时,采用S波数据反演$V_s$近似于单参数反演。

%(2)在采用反射波恢复中深部模型的中低波数时,基于DIW的将WERTI方法扩展到弹性介质中,导出了弹性WERTI方法的反演梯度。
(2)将基于DIW的WERTI方法扩展到弹性介质中,导出了弹性WERTI方法的反演梯度。
%地面数据分离与空间波场解耦可以将背景$V_p$与$V_s$的反演分解为两个独立的步骤,
%也即先利用PP波反演$V_p$,后利用PS波反演$V_s$。
%为了解决弹性介质中复杂的模式转换带来的P与S波间的相互干扰,
通过核函数分析,指出PP,SP,以及PS的互相关模式对$V_s$的反射波路径都有着较强的干扰作用,只有SS模式能够代表PS波中右半支的S波路径。

%为了解决复杂的模式转换带来的P与S波间的相互干扰,
(3)采用地面数据分离来分别提取PP与PS波的走时残差,采用空间分离的方式来对$V_s$的梯度进行预条件,
从而将$V_p$与$V_s$的反演分成独立的两个步骤来实施。分析并指出采用PS波成像界面反偏移恢复PS
反射波存在的问题,转而采用PP波成像界面+层剥离的方式来实现$V_s$的更新。

(4)在E-LSRTM中,
%由于其可以看作是线性化的EFWI问题,本文采用EFWI中同样的思路加入模式解耦来压制参数耦合。
针对该线性化的反问题,文中设计了不同于EFWI的反演策略来进一步压制参数耦合的影响。
参数耦合不强时采用模式解耦预条件的梯度进行双参数同时反演,
参数耦合强烈时,先反演$\delta V_s$,后反演$\delta V_p$可以进一步压制$\delta
V_p$反演中来自$V_s$的干扰。

\section{论文不足之处与下一步计划}
虽然利用模式解耦的工具对弹性波中的反演问题做了许多工作,但是作者认为其中仍然存在很多尚未深入研究之处,研究工作今后需要在以下几个方面展开:

(1)模式解耦实质上利用了Hessian的非对角块信息来加速收敛,但是Hessian对角块的信息利用并不完整,
因此在反演中如何与l-BFGS等优化方法结合进而更好地加速收敛仍待解决;

(2)走时反演有助于降低反演的非线性程度,但是其精度并不高,也容易出现多解性问题,下一步工作可以考虑在E-WERTI之后加入波形匹配的ERWI反演,进一步
提高中低波数成分的反演精度;

(3)利用PS波成像界面反偏移对$V_p$速度的误差过于敏感,如何利用PS波成像来较为准确的反射波能量需要寻找相应的解决办法,进而才能较稳健地用波形匹配
进行ERWI;

(4)E-LSRTM中同样面临密度扰动反演的问题,在线性化的问题中模式解耦或许能够对密度界面的反演有好的帮助,因此在E-LSRTM中利用模式解耦来解决密度
扰动也需要重新

(5)本文算法都是在理论模型上进行实验,缺乏对模式解耦,DIW等算法的抗噪性实验,下一步工作中需要在实际数据中测试算法,检验其实际应用中的潜力。
