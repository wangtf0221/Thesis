\chapter{结论与展望}
\section{结论}
为了恢复弹性介质模型中不同波数成分,论文针对弹性波反演问题,分别从EFWI,E-WERTI以及E-LSRTM三个方面入手来展开研究。
以模式解耦作为获取数据子集以及预条件梯度的主要手段,对以上三种反问题各自存在的挑战设计了不同的
反演策略,从而降低反问题的非线性程度和参数间的trade-off效应。通过文中的研究,主要得到了以下几点认识与结论:

(1)在EFWI多参数反演中,不同参数扰动产生的相同模式的数据扰动会造成参数耦合效应。
采用弹性波模式解耦对EFWI梯度进行预条件可以压制EFWI中的参数耦合效应。该预条件可以通过正传波场与解耦之后反传波场的互相关快速实现。
Hessian与分辨率矩阵的分析表明采用S波数据对$V_s$进行反演,在初始模型足够好的条件下可视为单参数反演,
同时也表明解耦正传波场对梯度进行预条件对进一步压制EFWI中的参数耦合贡献不大。

(2)含密度的三参数反演会增加EFWI的非线性程度,尤其是在软海底环境中PS波非常弱的情况下。模式解耦对压制密度与速度($V_p$和$V_s$)间的耦合作用有限,密度的反演仍然是
十分困难的课题,需要更多地加入多尺度策略甚至利用Hessian的信息。但即使引入密度变化,模式解耦对压制$V_p$与$V_s$间的参数耦合仍然有效。

(3)WERTI采用走时残差作为目标函数,与模型背景速度间存在更加线性的关系。相比互相关提取走时的方式,DIW方式在模型复杂的情况下更加可靠稳健。

(4)在反射波波路径中,PP,SP以及PS模式都对$V_s$梯度中的反射波波路径有着较强的干扰,只有SS模式能够代表PS波中右半支的S波路径。
因此地面P/S数据分离与空间波场模式解耦可以将背景$V_p$与$V_s$的反演分解为两个独立的步骤,即先利用PP波反演$V_p$,后利用PS波反演$V_s$。

(5)在E-LSRTM中Born模拟数据并不能直接匹配反射波数据,尤其是在偏移模型中包含较多高频信息时,需要匹配剩余反射数据才能与满足Born近似
下的数据逐步拟合。E-LSRTM可以提高成像分辨率并部分压制参数耦合,但是无法完全压制。模式解耦的预条件方式可以非常有效地压制$\delta
V_s$反演中来自$\delta V_p$的串扰。
在$\delta V_p$受参数耦合影响较弱时,在模式解耦帮助下进行双参数同时反演可以快速有效地获得$\delta
V_s$与$\delta V_p$。而在参数耦合程度较强时,可以通过先反演$\delta V_s$后反演$\delta
V_p$来进一步压制$\delta V_p$中来自$\delta V_s$的串扰。
\section{成果与创新点}
本文以模式解耦为核心搜段,针对三种弹性波反演方法(全波形反演,波动方程反射走时反演和最小平方偏移),提出了新的方法与反演策略。创新点主要有以下几个方面:

(1)提出了弹性波模式解耦全波形反演方法,
采用解耦反传波场而非地面数据来对梯度进行预条件,明显降低了$V_p$与$V_s$之间的串扰并提高了收敛效率。
从模式解耦出发分析相应的Hessian矩阵以及分辨率矩阵,阐明了它压制参数trade-off效应的内在机制。发现在初始模型足够好时,采用解耦出S波数据反演$V_s$近似于单参数反演。

%(2)在采用反射波恢复中深部模型的中低波数时,基于DIW的将WERTI方法扩展到弹性介质中,导出了弹性WERTI方法的反演梯度。
(2)发展了基于动态图像识别的弹性波波动方程反射走时反演方法。
%地面数据分离与空间波场解耦可以将背景$V_p$与$V_s$的反演分解为两个独立的步骤,
%也即先利用PP波反演$V_p$,后利用PS波反演$V_s$。
%为了解决弹性介质中复杂的模式转换带来的P与S波间的相互干扰,
通过核函数分析指出在$V_s$梯度计算中,PP、PS以及SP模式都会对$V_s$的反演带来干扰,只有SS模式能够代表PS波中右半支的S波路径。由此提出采用波场空间模式解耦预条件$V_s$
梯度从而压制复杂波路径造成的干扰。


%为了解决复杂的模式转换带来的P与S波间的相互干扰,
(3)采用地面数据P/S分离来分别提取PP与PS波的走时残差,采用波场空间模式解耦对$V_s$的梯度进行预条件,
将$V_p$与$V_s$的反演分成独立的两个步骤从而降低反演的非线性程度。发现采用PS波成像界面反偏移恢复PS
反射波存在的问题,提出了采用PP波成像界面+“层剥离”的方式来实现$V_s$更新的策略。

(4)在线性反问题弹性波最小平方偏移中,
%由于其可以看作是线性化的EFWI问题,本文采用EFWI中同样的思路加入模式解耦来压制参数耦合。
,设计了不同于EFWI的反演策略来进一步压制参数耦合的影响。
当参数耦合不强时采用模式解耦预条件的梯度进行双参数同时反演;
参数耦合强烈时,先反演$\delta V_s$,后反演$\delta V_p$,可以进一步压制$\delta
V_p$反演中来自$V_s$的串扰。

这些方法与策略形成了完整的弹性波反演流程以多尺度地恢复弹性模型的各类波数成分。

\section{论文不足之处与下一步计划}
虽然基于模式解耦对弹性波反演问题提出了一些有成效的方法与策略,但仍然存在很多尚待深入研究的问题。今后需要在以下几个方面展开进一步研究:

(1)模式解耦实质上利用了Hessian的非对角块信息来加速收敛。但是本文PCG算法框架中对Hessian对角块的信息利用并不充分。
在反演中如何与l-BFGS等优化方法结合进而更好地加速收敛仍待解决;

(2)走时反演有助于降低反演的非线性程度,但是其精度并不高,也存在多解性。下一步工作可以考虑在E-WERTI之后加入波形匹配的ERWI反演,进一步
提高中低波数成分的反演精度;

(3)利用PS波成像界面反偏移重建的反射波走时对$V_p$速度的误差过于敏感。如何利用PS波成像来较为准确地反偏移获得PS反射波的振幅仍需要寻找合适的解决办法,才能较稳健地用波形匹配
实现ERWI;

(4)E-LSRTM同样面临密度扰动反演的问题。在线性化的问题中模式解耦或许能够对密度界面的反演有帮助。因此在E-LSRTM中利用模式解耦来解决密度
扰动也需要重新审视与研究;

(5)本文算法都是在理论模型上进行实验,缺乏对模式解耦、DIW等算法的抗噪性实验。需要利用实际地震数据测试来检验本文方法的应用潜力。
