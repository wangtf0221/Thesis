\chapter{结论与展望}
\section{结论}
为了恢复地下介质模型中的不同波数成分,本文针对弹性波反演问题,分别从EFWI,EWERTI以及ELSRTM三个方面入手来展开研究。
以模式解耦作为获取数据子集以及预条件梯度的主要手段,对以上三种反问题各自存在的挑战设计了不同的
反演策略,从而降低反问题的非线性程度和参数间的trade-off效应。通过文中的研究,主要得到了以下几点认识与结论:

(1)在EFWI多参数反演中,不同参数扰动产生的相同模式的数据扰动会造成参数耦合效应。
采用弹性波模式解耦对EFWI梯度进行预条件有助于压制EFWI中的参数耦合效应。该预条件可以通过正传波场与解耦之后反传波场的互相关快速实现。
Hessian与分辨率矩阵的分析表明采用S波数据对$V_s$进行反演,在初始模型足够好时可视为单参数反演,
同时也发现解耦正传波场对梯度进行预条件对进一步压制EFWI中的参数耦合贡献不大。

(2)含密度的三参数反演会增加EFWI的非线性程度,尤其是在软海底环境中PS波非常弱的情况下。模式解耦对压制密度与速度($V_p$和$V_s$)间的耦合作用非常有限,密度的反演仍然是
十分困难的课题,需要更多地加入多尺度策略甚至利用Hessian的信息。但即使引入密度参数,模式解耦对压制$V_p$与$V_s$间的耦合效应仍然有效。

(3)WERTI采用反射走时残差作为目标函数,与模型背景速度间的关系更趋于线性化。相比互相关提取走时的方式,DIW方式在模型复杂的情况下更加稳健可靠。

(4)在反射波波路径中,PP、SP以及PS模式都对$V_s$梯度中的反射波波路径有着较强的干扰,只有SS模式能够代表PS波右半支的S波路径。
因此地面P/S数据分离与空间波场模式解耦可以帮助将背景$V_p$与$V_s$的反演分解为两个独立的步骤,即先利用PP波反演$V_p$,后利用PS波反演$V_s$。

(5)
ELSRTM可以提高成像分辨率并部分压制参数耦合效应,但是不太彻底。模式解耦的预条件方式可以非常有效地压制$\delta
V_s$图像中来自$\delta V_p$的串扰。
在$\delta V_p$受参数耦合影响较弱时,采用模式解耦预条件进行双参数同时反演可以快速有效地获得$\delta
V_s$与$\delta V_p$图像。而在参数耦合程度较强时,可以通过模式解耦预条件先反演$\delta V_s$后反演$\delta
V_p$来进一步压制$\delta V_p$中来自$\delta V_s$的串扰。
在ELSRTM中Born模拟数据与反射波数据很难匹配,尤其是在偏移模型中包含较多高频散射波时,需要匹配剩余反射数据才能与基于Born近似
的正演数据拟合。
\section{成果与创新点}
本文以模式解耦为核心手段,针对三种弹性波反演方法(即全波形反演,波动方程反射走时反演和最小平方偏移),提出了新的方法与反演策略。创新点主要有以下几个方面:

(1)提出了弹性波模式解耦全波形反演方法,
采用解耦反传波场而非地面数据来对梯度进行预条件,明显降低了$V_p$与$V_s$之间的串扰并提高了收敛效率。
%并指出在初始模型足够好时,采用解耦的S波数据反演$V_s$近似于单参数反演。
并基于模式解耦的Frechet导数、Hessian与分辨率矩阵,阐明了压制参数trade-off的内在机制。
%从模式解耦出发分析相应的Hessian矩阵以及分辨率矩阵,阐明了它压制参数trade-off效应的内在机制。发现在初始模型足够好时,采用解耦出S波数据反演$V_s$近似于单参数反演。

%(2)在采用反射波恢复中深部模型的中低波数时,基于DIW的将WERTI方法扩展到弹性介质中,导出了弹性WERTI方法的反演梯度。
%(2)发展了基于DIW的弹性波波动方程反射走时反演方法。
%核函数分析表明,PP、PS以及SP模式都会对$V_s$的反演带来干扰,只有SS模式能够代表PS波中右半支的S波路径。由此提出采用波场空间模式解耦预条件$V_s$
%梯度来压制干扰。
(2)通过反射敏感核函数分析发现空间波模式解耦预条件可以分离出
SS模式来计算PS波右半支S波路径,从而压制PP、PS以及SP模式对$V_s$反演带来的干扰。
提出采用地面数据P/S分离帮助获取PP和PS波的反射走时,由此发展了弹性波反射走时反演,
将$V_p$与$V_s$反演分成独立的两个步骤来降低非线性程度。
%指出采用PS波成像界面反偏移恢复PS
%反射波存在的问题,提出了采用PP波成像界面+“层剥离”的方式来实现$V_s$更新的策略。

(3)在弹性波最小平方偏移中,
%由于其可以看作是线性化的EFWI问题,本文采用EFWI中同样的思路加入模式解耦来压制参数耦合。
设计了不同于EFWI的反演策略来进一步压制参数耦合的影响。
当参数耦合不强时,采用模式解耦预条件的梯度进行双参数同时反演;
参数耦合强烈时,先反演$\delta V_s$,后反演$\delta V_p$,可以进一步压制$\delta
V_p$图像中来自$V_s$的串扰。

这些方法与策略形成了比较完整的弹性波反演流程,可以多尺度地恢复弹性介质纵横波速度的各类波数成分。

\section{论文不足之处与下一步计划}
虽然基于模式解耦对弹性波反演问题提出了一些有成效的方法与策略,但仍然存在很多尚待深入研究的问题。今后需要在以下几个方面展开进一步研究:

(1)模式解耦实质上利用了Hessian的非对角块信息来加速收敛。但是本文PCG算法框架中对Hessian对角块信息的利用并不充分。
在反演中如何与l-BFGS等优化方法结合进而更好地加速收敛仍待研究;

(2)走时反演有助于降低反演的非线性程度,但是其分辨率并不高,也存在多解性。下一步工作可以考虑在EWERTI之后加入波形匹配的ERWI反演,进一步
提高中低波数成分的反演精度;

(3)利用PS波成像界面反偏移重建的反射波走时对$V_p$速度的误差过于敏感。如何利用PS波成像来较为准确地反偏移获得振幅精度较高的PS反射波
仍需要寻找合适的解决办法,才能较稳健地用波形匹配
实现ERWI;

(4)ELSRTM同样面临密度扰动反演的问题。在线性化的反问题中模式解耦或许能够对密度界面的反演有帮助。因此在ELSRTM中利用模式解耦来解决密度
扰动也需要重新审视与研究;

(5)本文算法都是在理论模型上进行分析,缺乏对模式解耦、DIW等算法的抗噪性实验。需要利用实际地震数据测试来检验本文方法的实用性。
