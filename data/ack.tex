
%%% Local Variables:
%%% mode: latex
%%% TeX-master: "../main"
%%% End:

%\chapter{致谢}
\begin{ack}
	朝雨浥去轻尘,窗外柳色正新。樱花第十次绽放,爱校路又一次挤满了即将远行的留影人。只是这一季的毕业,终于到了我们。
	十年前,弱冠懵懂,踏着同济百年校庆的灿烂而来;十年后,已近而立,见证了同济十年的辉煌。
	春夏秋冬四季景,印在脑海里的是美好的记忆;本硕博十载青春,放在时光里的是道不尽的感激与不舍。
	如今即将博士毕业,回首在上海的这十年,同济俨然已经成了第二个家。
	从新奇到熟悉,从熟悉到留恋,而这一次的离开又如同记忆中十八岁时的远行,留恋却又无奈。
	樱花灿烂,终会凋零;胜地不常,盛筵难再。有离别才会显得相聚的珍贵,人生总是在不停的变换中才会丰富多彩。

	在这里我首先想感谢我的导师程玖兵教授。您教我成长,诲我做人,从大三至今已将有八年矣。
	从概念到理论,从编程到撰文,您在科研中每一处细节都用精益求精的态度来要求自己和我们。
	解惑时您耳提面命,懈怠时您谆谆教诲,迷茫时您指点迷津,如今博士论文已行将完成,其中每一页里面都包含了您的心血与付出。
	明师之恩,重于父母多矣,无以为报,当以更大的努力向您看齐。

	地震组大家庭是我梦想启航的地方,从这里收获的知识与技能让我有信心奔向远方。马在田院士蜚声海外,他的努力让今天的地震组人才济济,先生之风,山高水长。
	耿建华教授在储层地球物理课上悉心指导,王华忠教授在成像与反演以及信号处理中耐心研讨,董良国教授在地震波传播与成像方面开导启蒙,
	杨锴教授在地震勘探原理细心讲解,钟广法和刘堂晏教授在地震解释、测井以及岩石物理方面尽心辅导,
	刘玉柱教授在反演理论中指点迷津,赵峦啸和屠宁老师在组会讨论中指点帮助。衷心感谢各位老师传道育人,让我们这些学子终身受益!

	师兄弟姐妹的同门之谊亦令我终生难忘。特别感谢王晨龙博士,十年里学习、科研与生活的相伴,手足之情无过于此。
	感谢已毕业的三位硕士师兄滕龙、康玮、段鹏飞,你们留下的丰厚研究成果让我们受益良多。感谢尚颖霞、
	徐文才、杨涛、杨亚丽、邹鹏、阮晟、于洋、阮伦一众师弟师妹在生活与学业上的无私帮助。
	此外,感谢一众博士师兄王雄文、王义、王毓伟、杨积忠、迟本鑫、黄超、孙维蔷、周洋,与你们在科研上的讨论使我少走了许多弯路。
	感谢孙敏敖、于鹏飞、王红涛、杨靖康博士和其他兄弟姐妹的无私慷慨的帮助。

	最后,我要感谢父母的养育之恩、长兄的手足之情,感谢亲朋好友在读博期间的支持。
	我更要感谢我的未婚妻张悦。此生幸运,在最美好的年华里遇到你;夫复何求,在剩下的人生中与你相濡以沫、风雨相伴。
	谨以此文献给你们!
	
	\rightline{王腾飞于二零一七年樱花盛开之际}
%	\rightline{于二零一七年清明}



\end{ack}
