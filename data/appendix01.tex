\chapter{Gauss-Newton梯度的调查}
\label{cha:InvestigationOfGNGradient}
利用方程\eqref{eq:GN}和\eqref{eq:HessInv}, 可以得到GN方法预条件后的梯度:
\begin{equation}
        \delta \tilde{\mathbf{m}}=
        \begin{bmatrix}
                \delta \mathbf{V}_p\\
                \delta\mathbf{V}_s
        \end{bmatrix}
        =-
        \begin{bmatrix}
                \mathbf{D}\mathbf{g}^P_{V_p} +
                \mathbf{E}(\mathbf{g}^P_{V_s}+\mathbf{g}^S_{V_s})\\
                (\mathbf{F}\mathbf{g}^P_{V_p}+\mathbf{G}\mathbf{g}^P_{V_s}) +
                \mathbf{G}\mathbf{g}^S_{V_s}
        \end{bmatrix}.
        \label{eq:PreGN}
\end{equation}
注意到上式中利用了关系$\mathbf{g}_{V_p}=\mathbf{g}^P_{V_p}+\mathbf{g}^S_{V_p}$,
$\mathbf{g}_{V_s}=\mathbf{g}^P_{V_s}+\mathbf{g}^S_{V_s}$和$\mathbf{g}_{V_p}=\mathbf{g}^P_{V_p}$。很自然地,同样也
可以将预条件后的梯度$\delta\tilde{\mathbf{m}}$,分成两个部分:
$\delta \tilde{\mathbf{m}} = \delta \tilde{\mathbf{m}}^P+\delta
\tilde{\mathbf{m}}^S$, 其中
\begin{equation}
        \delta \tilde{\mathbf{m}}^P=
        \begin{bmatrix}
                \delta \mathbf{V}^P_p\\
                \delta \mathbf{V}^P_s
        \end{bmatrix}
        =-
        \begin{bmatrix}
                \mathbf{D}\mathbf{g}^P_{V_p} +
                \mathbf{E}\mathbf{g}^P_{V_s}\\
                \mathbf{F}\mathbf{g}^P_{V_p}+\mathbf{G}\mathbf{g}^P_{V_s}
        \end{bmatrix},
        \label{eq:PreGNP}
\end{equation}
和
\begin{equation}
        \delta \tilde{\mathbf{m}}^S=
        \begin{bmatrix}
                \delta \mathbf{V}^S_p\\
                \delta \mathbf{V}^S_s\\
        \end{bmatrix}
        =-
        \begin{bmatrix}
                \mathbf{E}\mathbf{g}^S_{V_s}\\
                \mathbf{G}\mathbf{g}^S_{V_s}.
        \end{bmatrix}
        \label{eq:PreGNS}
\end{equation}
考虑到$\delta \tilde{\mathbf{m}}^P=\mathbf{R}^P\delta \mathbf{m}$ 以及$\delta \tilde{\mathbf{m}}^S=\mathbf{R}^S\delta \mathbf{m}$,则
\begin{equation}
\begin{split}
        &\delta \mathbf{V}^P_s\approx0, \\
        &\delta \mathbf{V}_s \approx \delta
        \mathbf{V}^S_s=-\mathbf{G}\mathbf{g}^S_{V_s},
        \label{eq:gradGN}
\end{split}
\end{equation}
这是因为$\mathbf{R}^P$的底部区块几乎为空(见图\ref{fig:Resolution}b)

而由于$H^S_a=\mathbf{J}^{\dagger}\mathbf{J}^S\approx[\mathbf{J}^S]^{\dagger}\mathbf{J}^S$,则方程\eqref{eq:ResoOperS}变为
:
\begin{equation}
        \mathbf{R}^S\approx\mathbf{H}^{-g}_a[\mathbf{J}^S]^{\dagger}\mathbf{J}^S.
        \label{eq:ResoOperS1}
\end{equation}
将方程\eqref{eq:HessInv}带入到\eqref{eq:ResoOperS1}中,可以得到:
\begin{equation}
        \begin{split}
        \mathbf{R}^S
        &\approx
    \begin{bmatrix}
                \mathbf{D}&\mathbf{E} \\
                \mathbf{F}&\mathbf{G}
        \end{bmatrix}
    \begin{bmatrix}
        \mathbf{0}&\mathbf{0}\\
        \mathbf{0}&[\mathbf{J}^S_{V_s}]^{\dagger}\mathbf{J}_{V_s}^S
        \end{bmatrix}\\
        &=
    \begin{bmatrix}
                \mathbf{0}&\mathbf{E}[\mathbf{J}^S_{V_s}]^{\dagger}\mathbf{J}^S_{V_s}\\
                \mathbf{0}&\mathbf{G}[\mathbf{J}^S_{V_s}]^{\dagger}\mathbf{J}^S_{V_s}
        \end{bmatrix},
        \end{split}
        \label{eq:ResoOperS2}
\end{equation}
因为$\mathbf{J}^S=(\mathbf{0}\quad\mathbf{J}^S_{V_s})$。
注意到$[\mathbf{J}^S_{V_s}]^{\dagger}\mathbf{J}^S_{V_s}$代表了S波关于$V_s$扰动的Frech{$\acute{e}$}t导数自相关。正如图
\ref{fig:IllustrReso}所示,如果观测足够“完美”,则该分辨率矩阵的对角区块几乎是单位矩阵。这就意味着,$\mathbf{G}$近似地
代表了$\mathbf{J}^S_{V_s}]^{\dagger}\mathbf{J}^S_{V_s}$的逆。因此,预条件算子$\mathbf{G}$可以消除$V_s$ 反演中,单独采用S波
数据时几何扩散以及有限频带效应的影响。
