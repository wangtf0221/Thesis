\chapter{共轭状态法推导弹性波WERTI的梯度}
\label{cha:AdjointForEWERTI}
本节中,采用共轭状态法(Plessix(2006)\cite{plessix2006})导出弹性波WERTI的梯度。EWERTI中共有三个状态方程,其中两个为式\eqref{eq:WE}
和\eqref{eq:DeltaWE}。第三个状态方程来自DIW的优化约束。

我们知道DIW要找出局部的走时残差($\mathbf{\tau}=\tau(\mathbf{x}_r,t;\mathbf{x}_s)$)使得目标函数$D(\tau)$最小,其中$D(\tau)$为:
\begin{equation}
	D(\tau)=\frac{1}{2}\int
	d\mathbf{x}_sd\mathbf{x}_r(d^c(\mathbf{x}_r,t;\mathbf{x}_s)-
	d^o(\mathbf{x}_r,t+\tau(\mathbf{x}_r,t;\mathbf{x}_s);\mathbf{x}_s))^2
        \label{eq:Dl}
\end{equation}
而目标函数最小则需要满足:
\begin{equation}
	\frac{\partial D}{\partial \tau}=\int
	d\mathbf{x}_sd\mathbf{x}_r\alpha(\mathbf{x}_r,t;\mathbf{x}_s)=0.
        \label{eq:PartialD}
\end{equation}
其中$\alpha(\mathbf{x}_r,t;\mathbf{x}_s)=\dot{d}^o(\mathbf{x}_r,t+\tau;\mathbf{x}_s)(d^o(\mathbf{x}_r,t+\tau;\mathbf{x}_s)-
d^c(\mathbf{x}_r,t;\mathbf{x}_s))$。
偏导数$\frac{\partial D}{\partial
\tau}$一定为0,如果不为0,则总能找到一个$\tau$使得$D$变得更小,这与DIW找到的最小值相矛盾。因此,式\eqref{eq:PartialD}为
第三个状态方程。定义增广函数:
\begin{equation}
	\begin{split}
	\mathcal{L}(\mathbf{\tau},\mathbf{u},\mathbf{\psi},\mathbf{\mu},\bar{\mathbf{u}},\bar{\mathbf{\psi}})
	&=\frac{1}{2}\int \tau^2(\mathbf{x}_r,t;\mathbf{x}_s)d\mathbf{x}_sd\mathbf{x}_rdt\\
	&-\int
	\mu(\mathbf{x}_r,t;\mathbf{x}_s)\alpha(\mathbf{x}_r,t;\mathbf{x}_s)d\mathbf{x}_sd\mathbf{x}_rdt\\
	&-\int {\delta \psi_i}\left(\rho\frac{\partial^2 u_i }{\partial
	t^2}-\frac{\partial}{\partial x_j}(c_{ijkl}\frac{\partial u_k}{\partial x_l}) -
	f_i\right)d\mathbf{x}_sd\mathbf{x}_rdt\\
	&-\int \psi_i\left(\rho\frac{\partial^2
	\delta u_i }{\partial t^2}-\frac{\partial}{\partial x_j}(c_{ijkl}\frac{\partial
	\delta u_k}{\partial x_l})-\frac{\partial}{\partial x_j}(\delta
	c_{ijkl}\frac{\partial u_k}{\partial x_l})\right)d\mathbf{x}_sd\mathbf{x}_rdt,
	\end{split}
        \label{eq:Lagrangian}
\end{equation}
其中,$\tau(\mathbf{x}_r,t;\mathbf{x}_s)$,$\mathbf{u}$和$\delta \mathbf{u}$为状态变量,同时$\mu(\mathbf{x}_r,t;\mathbf{x}_s)$, 
$\boldsymbol{\psi}$和$\boldsymbol{\delta\psi}$为伴随状态变量。后两者也即反传波场。

伴随状态变量可以通过令增广函数对状态变量偏导数为零来求取,即使得$\frac{\partial \mathcal{L}}{\partial \mathbf{\tau}}=0$,
$\frac{\partial \mathcal{L}}{\partial \delta u}=0$和$\frac{\partial \mathcal{L}}{\partial 
	u}=0$。
首先
\begin{equation}
	\frac{\partial \mathcal{L}}{\partial \mathbf{\tau}}=\tau(\mathbf{x}_r,t;\mathbf{x}_s)
	-\mu(\mathbf{x}_r,t;\mathbf{x}_s)\frac{\partial \alpha(\mathbf{x}_r,t;\mathbf{x}_s)}{\partial \tau(\mathbf{x}_r,t;\mathbf{x}_s)},
        \label{eq:PartialTau}
\end{equation}
可得
\begin{equation}
	\mu(\mathbf{x}_r,t;\mathbf{x}_s)=\frac{\tau(\mathbf{x}_r,t;\mathbf{x}_s)}{ h_i(\mathbf{x}_r,t;\mathbf{x}_s)},
        \label{eq:AdjointTau}
\end{equation}
其中$h_i(\mathbf{x_r},t;\mathbf{x_s})=(\dot{d}^o_i(\mathbf{x_r},t+\tau;\mathbf{x_s}))^2-\ddot{d}^o_i(\mathbf{x_r},t+\tau;\mathbf{x_s})
(d^c_i(\mathbf{x_r},t;\mathbf{x_s})-d^o_i(\mathbf{x_r},t+\tau;\mathbf{x_s}))$, $\dot{d}$代表时间方向导数。

其次:
\begin{equation}
	\frac{\partial \mathcal{L}}{\partial \delta u}=
	\rho\frac{\partial^2 {\psi}_i }{\partial t^2}
	-\frac{\partial}{\partial x_j}(c_{ijkl}\frac{\partial {\psi}_k}{\partial x_l})-\mu(\mathbf{x}_r,t;\mathbf{x}_s)
	\dot{d}^o_i(\mathbf{x_r},t+\tau;\mathbf{x_s}).
%	\frac{\partial}{\partial x_j}(\delta c_{ijkl}\frac{\partial \bar{u}_k}{\partial x_l}).
        \label{eq:Partial_U}
\end{equation}
注意到这里利用反射数据的走时残差,因此$\mathbf{d}^c=\mathcal{F}(\delta
\mathbf{u})$,所以上式中对$\delta u$求导时需要加入DIW的约束。
于是,背景波场的伴随状态方程为:
\begin{equation}
	\rho\frac{\partial^2 {\psi}_i }{\partial t^2}
	-\frac{\partial}{\partial x_j}(c_{ijkl}\frac{\partial {\psi}_k}{\partial x_l})=\mu(\mathbf{x}_r,t;\mathbf{x}_s)
	\dot{d}^o_i(\mathbf{x_r},t+\tau;\mathbf{x_s}).
        \label{eq:Adjoint_deltaU}
\end{equation}

同上利用$\frac{\partial \mathcal{L}}{\partial u}=0$可得散射波场的伴随状态方程:
\begin{equation}
	\rho\frac{\partial^2 \delta {\psi}_i }{\partial t^2}
	-\frac{\partial}{\partial x_j}(c_{ijkl}\frac{\partial \delta {\psi}_k}{\partial x_l})=
	\frac{\partial}{\partial x_j}(\delta c_{ijkl}\frac{\partial {\psi}_k}{\partial x_l}).
        \label{eq:Adjoint_U}
\end{equation}
利用式\eqref{eq:Adjoint_deltaU}和\eqref{eq:Adjoint_U}求出伴随状态变量后就可以求得梯度:
\begin{equation}
	\frac{\partial \mathcal{L}}{\partial c_{ijkl}}=
    \frac{\partial E}{\partial c_{ijkl}}=-\int (\frac{\partial u_{i}}{\partial
    x_j}\frac{\partial \delta{\psi}_{k}}{\partial x_l}+\frac{\partial \delta {u}_{i}}{\partial
    x_j}\frac{\partial \psi_{k}}{\partial x_l}).
    \label{eq:GradientCijkl1}
\end{equation}

