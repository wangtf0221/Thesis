
%%% Local Variables:
%%% mode: latex
%%% TeX-master: t
%%% End:



\chapter{弹性波波动方程反射FWI走时反演}

\section{引言}
在第1章中,我们知道
EFWI可以提供高精度地下模型参数估计,但是需要有非常好的初始模型。走时反演只利用了走时残差,而走时残差与低波数成分的模型扰动具有更强的线性关系,所以这两者的联系
会更敏感。因此,采用弹性波波动放程反射走时反演(WERTI)来建立P和S波速度的初始模型将会更加稳健且有效,尤其是对深部的宏观结构恢复更有帮助。在弹性介质中,为了区
分P和S波的反射走时,我们将地面地震数据分解为矢量的P-和S-波地震记录。反射P波与S波的走时信息采用Dynamic
image
warping(DIW)的方式来获取。我们首先采用P波走时来反演P波速度,其次采用S波走时来反演S波速度。
在此过程中,我们采用模式解耦来求取预条件的梯度。Sigsbee2A模型的数值实验例子证明了我们的方法可以有效得获取模型长波长信息。

\subsection{引言}
弹性波全波形反演可以提供高精度的地下模型参数估计,但是也同样受困于多参数反演中的强烈非线性程度以及声波情况下的周波跳跃问题\cite[]{sears:2008,brossier2009}。
Xu et al. (2012)\cite{xu:2012}提出一种反射波波形反演(RWI)的方法,其通过采用偏移/反偏移的方式来预测反射波,从而估计模型中的长波长成分。采用拟合波形的思路,近期
Wu and Alkhalifah (2015)\cite[]{Wu2015b}以及Zhou et al. (2015)\cite[]{Zhou2015}发展了该RWI方法,而Guo and Alkhalifah (2016)\cite{Guo2016}则将该方法扩展到了
弹性介质中。

尽管波形信息能提高反演精度,但是走时信息对模型的低波数成分扰动更敏感,二者之间具有更好的线性关系。所以,采用走时反演来为常规FWI建立合适的含有长波长信息的初始
模型会更加稳健有效\cite[]{Ma2013, Chi2015, Luo2016}。然而在弹性介质中,这中思路会受到限制,因为弹性波场中存在复杂的模式转换,这就导致很难获得单一波模式的走时
信息。在本文中,我们将致力于通过波模式解耦和“动态图像打开”(DIW)\cite[]{Ma2013}来区分出PP和PS的反射走时。之后,我们利用区分出的PP和PS反射走时,通过一个两步
的反演流程来实现WERTI\cite[]{Hale2013}。最后,最后,Sigsbee2A模型的数值实验结果证明了弹性波WERTI方法的有效性。
\subsection{弹性WERTI方法的目标函数和梯度}
假设在背景弹性介质$c_{ijkl}$中有一个参数扰动$\delta c_{ijkl}$,则背景波场与扰动波场满足以下方程:
\begin{equation}
    \rho \frac{\partial u^2_i}{\partial t^2}  -
    \frac{\partial}{\partial x_j}\left[ 
        c_{ijkl}\frac{\partial u_{k}}{\partial
        x_l}\right]=f_i,
    \label{eq:WE} 
\end{equation}
和
\begin{equation}
    \rho \frac{\partial \delta u^2_i}{\partial t^2}  -
    \frac{\partial}{\partial x_j}\left[ 
        c_{ijkl}\frac{\partial \delta u_{k}}{\partial
        x_l}\right]=\frac{\partial}{\partial x_j}\left[\delta c_{ijkl}\frac{\partial u_{k}}{\partial x_l}\right],
    \label{eq:DeltaWE} 
\end{equation}
其中,$\delta u_i$可以看作是用RTM或其他方法的成像结果,$\delta c_{ijkl}$进行反偏移之后的反射数据。在WERTI中,我们需要使得观测数据$\mathbf{d}^{o}$与模拟数据
$\mathbf{d}^{c}$之间的走时残差最小,则目标函数为:
\begin{equation}
    E=\frac{1}{2}\int\tau^2(\mathbf{x_r},t;\mathbf{x_s})dtd\mathbf{x_r}d\mathbf{x_s},
    \label{eq:Objectivefunction} 
\end{equation}
其中,走时残差$\tau(\mathbf{x_r},t;\mathbf{x_s})$可以通过DIW来获取。通过类似Ma and
Hale (2013)\cite{Ma2013}的推导(见附录\ref{cha:AdjointForEWERTI}),我们可以得到如下梯度公式:
\begin{equation}
    \frac{\partial E}{\partial c_{ijkl}}=-\int (\frac{\partial u_{i}}{\partial
    x_j}\frac{\partial \delta \psi_{k}}{\partial x_l}+\frac{\partial \delta u_{i}}{\partial
    x_j}\frac{\partial \psi_{k}}{\partial x_l}),
    \label{eq:GradientCijkl}
\end{equation}
其中,$\psi_i$和$\delta \psi_i$是共轭波场满足:
\begin{equation}
    \rho \frac{\partial \psi^2_i}{\partial t^2}  -
    \frac{\partial}{\partial x_j}\left[ 
        c_{ijkl}\frac{\partial \psi_{k}}{\partial
        x_l}\right]=\tau(\mathbf{x_r},t;\mathbf{x_s})\frac{\dot{d}^o_i(\mathbf{x_r},t+\tau;\mathbf{x_s})}{h_i(\mathbf{x_r},t;\mathbf{x_s})},
    \label{eq:AdjointWE} 
\end{equation}
和
\begin{equation}
    \rho \frac{\partial \delta \psi^2_i}{\partial t^2}  -
    \frac{\partial}{\partial x_j}\left[ 
        c_{ijkl}\frac{\partial \delta \psi_{k}}{\partial 
        x_l}\right]=\frac{\partial}{\partial x_j}\left[\delta c_{ijkl}\frac{\partial
        \psi_{k}}{\partial x_l}\right], 
    \label{eq:AdjointDeltaWE} 
\end{equation}
上式中$h_i(\mathbf{x_r},t;\mathbf{x_s})=(\dot{d}^o_i(\mathbf{x_r},t+\tau;\mathbf{x_s}))^2-\ddot{d}^o_i(\mathbf{x_r},t+\tau;\mathbf{x_s})
(d^c_i(\mathbf{x_r},t;\mathbf{x_s})-d^o_i(\mathbf{x_r},t+\tau;\mathbf{x_s}))$,而$\dot{d}$则代表$d$在时间方向的导数。在方程\eqref{eq:GradientCijkl}的右端项
中,第一和第二项分别表示震源端和检波点端的反射波路径。之后我们可以通过链式法则来导出P波与S波速度的梯度公式:
\begin{equation}
    \frac{\partial E}{\partial V_p}=2\rho V_p\frac{\partial E}{\partial
        c_{ijkl}}\delta_{ij}\delta_{kl}, \qquad
    \frac{\partial E}{\partial V_s}=2\rho V_s\frac{\partial
    E}{\partial c_{ijkl}}(-2\delta_{ij}\delta_{kl}+\delta_{ik}\delta_{jl}+
    \delta_{il}\delta_{jk}).
    \label{eq:GradientVel}
\end{equation}
\subsection{弹性波WERTI工作流程}
弹性波介质中,不同模式的转换波,主要是PP与PS同相轴之间,会互相叠加、互相交叉。在用DIW提取走时差的时候,这些交叉点的位置就会变成一些奇异点而给拾取带来很大困难
。因此,如果直接用原始多分量数据的话,拾取到的$\tau(\mathbf{x_r},t;\mathbf{x_s})$就会不准确,这也是为什么式\eqref{eq:AdjointDeltaWE}中的梯度很难直接应用。
为了解决这个问题,我们将观测与模拟地震数据分解为P波与S波两部分。该模式分解只作用于地面地震数据\cite[]{Li2016a},因此我们可以快速地获得每一炮数据分解之后的矢量
P波或S波地震记录。这样,走时残差就可以被分成P波与S波两部分,使得弹性波WERTI变得可行,也即用一个两阶段的流程,先P波阶段后S波阶段。

在P波阶段,主要采用PP反射波来建立P波速度的背景模型。首先,我们从ERTM中获得P波速度的扰动,也即$\delta V_p$。由于在WERTI中我们只考虑走时,所以我们只进行ERTM成像
而不是最小平方ERTM来拟合反射波振幅。因此,目标函数可以变做最小化PP反射波的走时残差:
\begin{equation}
    E_{pp}=\frac{1}{2}\int\tau^2_{pp}(\mathbf{x_r},t;\mathbf{x_s})dtd\mathbf{x_r}d\mathbf{x_s}.
    \label{eq:ObjectivefunctionPP} 
\end{equation}
这样,采用P波记录来计算方程\eqref{eq:AdjointWE}的右端项,并用$\delta V_p$替换方程\eqref{eq:DeltaWE}和\eqref{eq:AdjointDeltaWE}中的$\delta c_{ijkl}$,我们
就可以计算得到$V_p$的梯度, $\frac{\partial E}{\partial V_p}$。

在S波阶段,我们将采用类似的策略但是只利用PS反射波。则目标函数变为:
\begin{equation}
    E_{ps}=\frac{1}{2}\int\tau^2_{ps}(\mathbf{x_r},t;\mathbf{x_s})dtd\mathbf{x_r}d\mathbf{x_s}.
    \label{eq:ObjectivefunctionPS} 
\end{equation}
但是,实现方式与前一阶段策略略有不同。经过P波阶段反演之后,$V_p$的背景速度已经获得了很好的恢复,此时PP波成像结果应该已经比较接近准确位置。而通常情况下,地
下介质中$V_p$与$V_s$的界面是比较一致的。所以,我们推荐采用较为准确的PP成像($\delta V_p$)结果而不是$\delta V_s$来产生PS反射波。此外,在PS反射中,波路径的
震源端只与P波速度相关,因此我们在计算$\frac{\partial E}{\partial V_s}$可以丢掉方程\eqref{eq:GradientCijkl} 右端项的第一部分。并且,波模式分解也会施加在
梯度的计算过程中来确保只有S波能量参与其中:
\begin{equation}
    \frac{\partial E_{ps}}{\partial V_s}=-2\rho V_s
    \int (\frac{\partial \delta u^S_{i}}{\partial
    x_j}\frac{\partial \psi^S_{k}}{\partial x_l})
    (\delta_{ik}\delta_{jl}+
    \delta_{il}\delta_{jk}).
    \label{eq:GradientVel}
\end{equation}
上述梯度与Wang et al. (2015)\cite{Wang2015a}提出的EFWI方法类似。模式解耦可以在$V_s$的反演中降低参数耦合同时也可以压制假象。

\subsection{局部倾角滤波约束}
WERTI非常类似成像域射线走时层析。而我们知道射线走时层析反问题通常不收敛,很难收敛或者很难收敛获得具有地质意义的模型。
在偏移后,成像结果可以提供地下界面大致的倾角信息,通过此类倾角信息,可以设计预条件算子来沿地层倾角来对反演进行约束,使得反演结果更合理。
采用倾角约束滤波可以加快收敛速度,在很大程度上改善反演结果。由于采用波动方程作为手段,WERTI采用波路径信息来反投影走时残差信息,因而其所面临的挑战要
小于射线走时层析。然而,对于一些照明不足,或者
反射路径覆盖不均匀的区域同样也会出现上述的问题。常规的各向同性的平滑算子很难保证反演收敛到具有地质含义的物理模型。因此,WERTI中采用局部倾角约束同样可以加速
收敛,并获得具有地质意义的反演结果。
