
%%% Local Variables:
%%% mode: latex
%%% TeX-master: t
%%% End:
\secretlevel{保密} 
\secretyear{2}

\ctitle{基于弹性波模式解耦的全波形反演方法}

% 按照申请工学学位设计。如有其它需要,请修改相应文字。
\makeatletter
  \iftongji@doctor
    \cdegree{博士}
  \else
    \iftongji@master
      \cdegree{理学硕士}
    \fi
  \fi

\makeatother

\cdepartment{海洋与地球科学学院}

\cmajorfirst{固体地球物理学}

\degtype{理学}

\cauthor{王腾飞}

\cstnr{1110701}

\csupervisor{程玖兵 教授}

% 如果没有副指导老师或者联合指导老师,把各自{}中内容留空即可。

\cassosupervisor{}

\ccosupervisor{}

% 日期自动生成,如果你要自己写就改这个cdate
%\cdate{\CJKdigits{\the\year}年\CJKnumber{\the\month}月}
\makeatletter
  \iftongji@doctor
    \edegree{Doctor of Philosophy}
  \else
    \iftongji@master
      \edegree{Master of Science}
    \fi
  \fi

\makeatother

%\cfunds{自然基金项目(No.123456789)}

%\efunds{(Supported by the Natural Science Foundation of China for\\ Distinguished
%         Young Scholars, Grant No.123456789)}

\etitle{Elastic full waveform inversion based on wave mode decomposition}

\edepartment{School of Ocean and Earth Science}

\edispline{Natural Science}

\emajorfirst{Solid Geophysics}
%\emajorsecond{TONGJILUG}

\eauthor{Tengfei Wang}

\esupervisor{Prof. Jiubing Cheng}

% 这个日期也会自动生成,你要改么?
% \edate{May, 2009}

% 定义中英文摘要和关键字
\begin{cabstract}
	通过地震数据定量地估计地下介质弹性参数甚至岩石物理参数是探测地球内部结构和勘探油气资源的核心任务。随着计算能力的快速提升以及
	长偏移距、宽方位、宽频带地震数据采集技术的应用,旨在估计全波数谱速度模型的全波形反演(FWI)方法
	正成为地震勘探中强有力的工具。然而在实际应用中,FWI往往无法获得理论上预期的能力。近二十年来,
	为了解决FWI受到周波跳跃(cycle-skipping)、子波估计不准以及噪音等问题的困扰,
	许多学者发展了分频率、分偏移
	距、分时窗、分散射角等多尺度策略来降低非线性程度。此外,在弹性介质中,
	不同波模式相互转换及多分量数据的其他复杂性
	进一步增加了反问题的非线性程度,而且
	不同参数扰动的偏导数波场在特定散射角范围内的重叠还会导致参数耦合效应。
	为此,本文围绕模式解耦这一数据分离工具,从弹性波全波形反演(EFWI)、弹性波波动方程反射走时反演
	(EWERTI)以及弹性波最小二乘逆时偏移(ELSRTM)出发,尝试恢复高、中、低各种波数成分的弹性参数模型,进而形成比较实用的弹性波
	反演方法与流程。

%	在第二章中主要讨论EFWI中模式解耦对压制参数耦合效应的作用。
	EFWI通过最小化观测到的多分量数据与正演模拟数据
	之间的残差来获得高分辨率的地下弹性参数模型。由于Hessian矩阵的显式计算与求逆代价十分
	巨大,实际应用中的大规模反演问题通常采用梯度类而非Hessian类的最优化方法来求解。然而,多参数
	反演中的参数耦合效应会引起不同参数间梯度的串扰,进而会严重影响反演的收敛速度与精度。
%	近期,弹性波模式解耦的方法被用来对梯度进行预条件从而降低EFWI过程中的参数耦合
%	问题。
	第二章提出了一种基于模式解耦的EFWI方法,通过时间域的正传
	波场与解耦的反传波场互相关来预条件梯度。文中基于解耦的Frech$\acute{e}t$导数来分析
	Hessian矩阵和分辨率矩阵的性态,并对比常规共轭梯度法、Gauss-Newton法和模式解耦法三者之间梯度的异同来解释模式解耦预条件处理压制参数耦合的物理机制。
	简单流体饱和模
	型与Marmousi-II模型数值实例证明了模式解耦的预条件共轭梯度法(MDPCG)可以降低P-波和S-波速度($V_p$和$V_s$)
	之间的参数耦合,在不涉及Hessian计算的情况下获得较高的收敛效率。

	弹性反射波波形反演(ERWI)有助于更新中深层模型的中低波数成分,从而为EFWI提供较好的初始模型。
	然而,波形匹配的ERWI同样面临周波跳跃的问题。相比波形匹配的目标函数,反射走时目标函数关于背景速度模型
	的非线性程度更低。因此第三章采用弹性波波动方程反射走时反演(EWERTI),
	通过DIW(Dynamic image warping)算法来提取走时残差并以此建立目标函数,
	可以一定程度上克服周波跳跃问题。
	地面多分量数据P/S分离可帮助获得不同模式的数据残差,空间域弹性波模式解耦则可以对梯度有效地预条件,从而实现$V_p$和$V_s$的分步反演。这有效降低了反问题的非线性程度。
	反射波波路径核函数分析解释了解耦不同波模式的反射路径对压制串扰的重要作用。
%	局部倾角导引正则化约束可以使得反演快速收敛到具有地质意义的模型。
	此外,
	为了降低反演的多解性,采用了成像剖面进行局部倾角导
	引正则化来确保反演获得具有地质意义的速度模型。
	也就是说,本文背景速度反演采用了两步策略,首先采用PP波数据反演$V_p$,然后固定反演好的$V_p$通过PS波反演$V_s$。
	然后,通过Sigsbee2A模型的数值实验结果
	验证EWERTI方法以及反演策略的有效性,并用EFWI检验EWERTI反演结果的准确性。

	ELSRTM旨在通过拟合反射波振幅信息来反演弹性参数模型的高波数成分,可视为线性的EFWI问题。
	常规ELSRTM通过多次迭代可以压制由于数据缺失、粗网格采样等产生的成像噪音,提高成像分辨率,同时部分地降低参数间的耦合
	效应。模式解耦则可以在计算梯度时通过分离出S波梯度,使得ELSRTM中S波速度扰动的估计接近于单参数反演,从而降低了反演中的参数耦合及非线性程度
	。于是,针对不同的参数耦合情况设计了相应
	的反演策略。在$V_p$较少受$V_s$耦合影响时,采用梯度解耦后的双参数同时反演;而在$V_p$与$V_s$强烈相互影响时,借助解耦后的S波
	反演$V_s$扰动,然后再反演$V_p$扰动,这样可以进一步压制$V_p$所受到的来自$V_s$的耦合影响。
%	另外,由于实际反射数据与Born模拟数据之间存在差异,因此在ELSRTM中最好能匹配反射数据与背景波场之间的残差。

%	为了恢复弹性介质模型中的不同波数成分,本文
	本文针对弹性波反演问题,分别从EFWI、EWERTI以及ELSRTM
	三个方面入手来恢复不同波数成分的弹性参数模型。在常规多尺度策略的基础上,通过空间域模式解耦和地面P/S分离
	获取P或S数据子集来适应反演中的不同需求。根据解耦波场数据在Frech$\acute{e}t$导数、
	Hessian和分辨率矩阵以及Born反射波路径计算中的贡献和影响,设计出适用于不同阶段的梯度预条件方法和多尺度策略,
	从而降低了弹性波反演的非线性和参数耦合程度,
	并形成了比较完整的弹性波反演方法系列,为从浅部到深部更准确地估计弹性参数提供了新的理论方法与技术支撑。


\end{cabstract}

\ckeywords{弹性波,模式解耦,全波形反演,Hessian和分辨率矩阵,参数耦合,反射走时反演,反射核函数,最小平方逆时偏移}

\begin{eabstract}
	The primary task of the detecting and imaging the Earth's interior is to estimate the elastic parameters or rock properties
	quantitively
	through the recording data on the surface. With the developement of 
	high-performance computation and the
	maturation of wide-azimuth, long-offset and broadband data accquisition 
	technology, full waveform inversion (FWI) becomes a powerful tool to recover the full
	 wavenumber spectrum of the subsurface. However, FWI can not rebuild the estimated
	model as good as expected. In order to overcome the obstacles such as cycle-skipping problems
	caused by bad initial models or
	insufficiency of low frequency components in the data, inaccurately estimated source
	wavelet, low sigal/noisy ratio and so on, many researchers developed hierarachical strategies
	by selecting data subsets of different frequency, offset or scattering-angle during the
	inversion.	
	In elastic media, complicated mode conversions and other multicomponent problems further increase the
	nonlinearity of inversion. Besides, in certain angles, the overlapped partial derivative wavefields of  different
	parameter will lead to trade-offs.
	To rebuild the elastic model of different wavenumber, elastic full waveform
	inversion (EFWI), elastic wave equation 
	reflection traveltime inversion (EWERTI) and elastic least-squares reverse time
	migration (ELSRTM) are implemented with the help of wave mode decomposition. Utilizing these
	methods, we
	try to recover the elastic model from low and itermediate wavenumber to high
	wavenumber, or even full wavenumber.

Elastic full waveform inversion (EFWI) aims to reduce the misfit between recorded and modelled multi-component
seismic data for deducing a detailed model of elastic parameters in the subsurface.
Because the explicit computation and inversion of the Hessian matrix
is extremely resource intensive,
a gradient-based (rather than Hessian-based) minimization is generally applied for large-scale applications.
However, the multi-parameter trade-off effects cause cross-talks in the computed gradients and
thus severely affect the convergence and the quality of the inverted model.
%Recently, preconditioning the gradients based on elastic wave mode decomposition
%has been suggested for mitigating the parameter trade-offs in the EFWI process.
In the second chapter, we propose a mode decomposition (MD)-based EFWI approach, in which the preconditioned gradients
are obtained through the cross-correlation of the forward and
decomposed adjoint wavefields in the time domain.
Based on the decomposed Frech{$\acute{e}$}t derivatives,
we explain the mechanism of this approach through analyses of Hessian and resolution matrices
and comparisons with the Gauss-Newton gradients.
Numerical examples of a simple fluid-saturated model and the Marmousi-II model
demonstrate that the MD-based preconditioned conjugate-gradient approach
can mitigate the trade-off between the P- and S-wave velocities and achieve fast
convergence without any Hessian-involved calculations.

Elastic reflection waveform inversion (ERWI) utilize the reflections to update the low and
intermediate wavenumber of
the deep part model, which can provide good initial models for EFWI. However, ERWI suffers from
the cycle-skipping problem due to the objective function of waveform fitting. Since taveltime
information relates to the background model more linearly, the cycle-skipping of traveltime
objective function will be less severe
compared with the previous one. Thus, in the third chapter we implement the WERTI by using the $L_2$ norm of the traveltime
residual extracted by the Dynamic image warping (DIW) as objective function.
The reflection kernel analysis shows that mode decomposition can suppress the artifacts in
gradient calculation. 
Besides, the model regularization through local dip-dependent smooth filter ensures the inversion converging to a
geological model. 
P/S
separation of multicomponent seismograms provides P or S recordings while spatial wave mode
decomposition provides P or S wavefields, which help
to  reduce the nonlinearty of inversion effectively.
Based on the above, a two-step inversion strategy is adopted, in which PP reflections are first used to invert $V_p$,
followed by $V_s$ inversion with PS reflections based on the well recoverd $V_p$. 
%The kernel of reflection wavepath proves that mode decomposition can surpress the artifacts in
%the reflection inversion. 
Numerical example of Sigsbee2A model validates the effectiveness of the
algorithms and strategies for elastic WERTI (E-WERTI).
%Numerical example of Sigsbee2A model validates the effectivenss of the 
%algorithms and strategies for EWERTI, whose results also are tested through EFWI.

ELSRTM is a linearized EFWI aimming to fit the waveform of reflections generated by the inverted
high-wavenumber perturbations. 
%Due to the energy difference between Born and reflection data, it is
%more suitable to match the residual between reflection data and the background data during ELSRTM.
LSRTM can reduce migration artifacts arising
from limited aperture, coarse sampling, and acquisition gaps. Conventional ELSRTM can sligthly but
not entirely mitigate the parameter trade-offs. 
Nonetheless, mode decomposition
isolates the S-wave part gradient, which makes the inversion of S-wave velocity
perturbation a mono-parameter inversion to help mitigate the trade-offs. In the fourth chapter, we
design different inversion strategies to cope with the different situations. When $V_p$ is little
affected by $V_s$, we invert the two parameter simultaneously with mode decomposition; when $V_p$
and $V_s$ couple with each
other severely, we recommand inverting $V_s$ firstly followed by $V_p$ inversion to further mitigate
the trade-offs from $V_s$ during the $V_p$ inversion.

In this thesis, we focus on the mode-decomposition-based inversion methods to recover the elastic model
containing different wavenumber. P/S separation of multicomponent data and wave mode
decomposition provide flexible P or S wave data in different stage of inversion.
The investigation of decomposed Frech{$\acute{e}$}t derivatives, Hessian and resolution matrices and Born reflection
kernels show the different contribution of P or S wave to gradients. Accordingly, hierarchical
strategies are designed to reduce the nonlinearty, trade-off
effects and other problems during inversion and finally rebuild the elastic model from shallow to
deep effectively. 

\end{eabstract}

%\ckeywords{全波形反演,Hessian和分辨率矩阵,参数耦合,波动方程反射走时反演,反射波路径核函数,最小平方逆时偏移,弹性波,模式解耦}
\ekeywords{Elasticity, mode decomposition, full waveform inversion, Hessian and resolution matrix,
	parameter trade-off, reflection kernel,
	wave equation reflection traveltime inversion, least-squares RTM,
	}
