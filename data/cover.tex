
%%% Local Variables:
%%% mode: latex
%%% TeX-master: t
%%% End:
\secretlevel{保密} 
\secretyear{2}

\ctitle{基于弹性波模式解耦的全波形反演方法}

% 按照申请工学学位设计。如有其它需要,请修改相应文字。
\makeatletter
  \iftongji@doctor
    \cdegree{博士}
  \else
    \iftongji@master
      \cdegree{理学硕士}
    \fi
  \fi

\makeatother

\cdepartment{同济大学海洋学院}

\cmajorfirst{固体地球物理}

\degtype{理学}

\cauthor{王腾飞}

\cstnr{1110701}

\csupervisor{程玖兵 教授}

% 如果没有副指导老师或者联合指导老师,把各自{}中内容留空即可。

\cassosupervisor{}

\ccosupervisor{}

% 日期自动生成,如果你要自己写就改这个cdate
%\cdate{\CJKdigits{\the\year}年\CJKnumber{\the\month}月}
\makeatletter
  \iftongji@doctor
    \edegree{Doctor of Philosophy}
  \else
    \iftongji@master
      \edegree{Master of Science}
    \fi
  \fi

\makeatother

%\cfunds{自然基金项目(No.123456789)}

%\efunds{(Supported by the Natural Science Foundation of China for\\ Distinguished
%         Young Scholars, Grant No.123456789)}

\etitle{Elastic Full waveform inversion based on wave mode decomposition}

\edepartment{School of Ocean and Earth Science}

\edispline{Natural Science}

\emajorfirst{Solid Geophysics}
%\emajorsecond{TONGJILUG}

\eauthor{Tengfei Wang}

\esupervisor{Prof. Jiubing Cheng}

%\eassosupervisor{Prof. Gang Wan}

% 这个日期也会自动生成,你要改么?
% \edate{May, 2009}

% 定义中英文摘要和关键字
\begin{cabstract}
	弹性波全波形反演方法(EFWI)想要通过最小化实际观测到的多分量数据与正演模拟的数据
	之间的残差来获得高分辨率的地下参数模型。由于Hessian矩阵的显式计算与反演代价十分
	巨大,大规模的应用问题通常采用梯度类的最优化方法而非基于Hessian的方法。然而多参数
	反演中,参数耦合效应会引起不同参数间梯度的串扰,进而会严重影响反演过程的速度与精度。
	近期以来,弹性波模式解偶的方法被用来对梯度进行预条件从而降低EFWI过程中的参数耦合
	问题。在本文中,我们提出了一种基于模式解偶(MD)的EFWI方法,该方法在时间域通过正传
	与解偶的反传波场之间的互相关来获得梯度。基于解偶的Frech$\acute{e}t$导数,我们通过分析
	Hessian矩阵和分辨率矩阵、对比Gauss-Newton梯度来解释本方法的物理机制。简单流体饱和模
	型与Marmousi-II模型的数值实例,证明了基于MD的预条件共轭梯度法可以降低P-波和S-波速度
	之间的参数耦合,并且在不涉及Hessian计算的同时获得快速的收敛。
\end{cabstract}

\ckeywords{\TeX, \LaTeX, CJK}

\begin{eabstract}
Elastic full waveform inversion (EFWI) aims to reduce the misfit between recorded and modelled multi-component
seismic data for deducing a detailed model of elastic parameters in the subsurface.
Because the explicit computation and inversion of the Hessian matrix
is extremely resource intensive,
a gradient-based (rather than Hessian-based) minimization is generally applied for large-scale applications.
However, the multi-parameter trade-off effects cause cross-talks in the computed gradients and
thus severely affect the convergence and the quality of the inverted model.
Recently, preconditioning the gradients based on elastic wave mode decomposition
has been suggested for mitigating the parameter trade-offs in the EFWI process.
In this paper, we propose a mode decomposition (MD)-based EFWI approach, in which the preconditioned gradients
are obtained through the cross-correlation of the forward and
decomposed adjoint wavefields in the time domain.
Based on the decomposed Frech{$\acute{e}$}t derivatives,
we explain the mechanism of this approach through analyses of Hessian and resolution matrices
and comparisons with the Gauss-Newton gradients.
Numerical examples of a simple fluid-saturated model and the Marmousi-II model
demonstrate that the MD-based preconditioned conjugate-gradient approach
can mitigate the trade-off between the P- and S-wave velocities and achieve fast
convergence without any Hessian-involved calculations.
\end{eabstract}

\ekeywords{Waveform inversion, Elasticity, Mode decomposition, Resolution matrix
}
