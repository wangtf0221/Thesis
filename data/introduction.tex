\section{引言}
\subsection{研究背景和意义}
地震勘探中深度域偏移成像质量非常依赖于速度模型的精度。且随着勘探的发展,成像技术中越来越需要考虑各向异性,而在各向异性介质中,成像则更加
依赖各向异性参数的准确性。因而需要考虑各向异性介质中的深度域建模。此外,基于层析的建模方法只能提供较好的偏移模型,但是地震勘探中我们总希
望得到对地下介质更为精细的刻画,这就需要引入全波形反演技术(FWI)。传统的FWI方法中通常基于拟声波方程来描述波场,这与地下介质的弹性假设并
不相符,近年来由于计算机能力提升、多分量观测数据的增多以及解决声波FWI无法回避的问题的需要,考虑弹性甚至各向异性的全波形反演逐渐成为研究热
点。弹性波多分量数据中同时含有纵波(P波)和横波(S波),这两种不同波模式对地下介质有着不同的刻画作用。近年来弹性波波模式分离技术能够提供准
确的P或S波数据,这将会对反演带来许多有益的帮助。此外,反射全波形反演(RWI)通过反射波信息来建立初始模型。对于弹性波RWI,模式解耦提供的分
离的P或S波数据同样会非常有帮助。
\subsection{研究现状}
\subsection{研究内容}
